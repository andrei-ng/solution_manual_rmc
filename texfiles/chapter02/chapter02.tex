\chapter[Chapter 2]{Solutions}

\subsection*{2-1}

It should be clear that the dot product of two vectors is independent of the reference frame used since the dot product can also be computed from $v_1 \cdot v_2 = |v_1| |v_2| \cos\theta$, i.e., it depends only on the magnitude of the two vectors and the angle between them.
\\~


Assume $v_1$, $v_2\space$ are vectors in $\R^3$. Introduce the notation $a=v_1$ and $b=v_2$.


Let $a^1$ and $b^1$ be the representation of the two vectors in frame $o_1x_1y_1z_1$ in $\R^3$.
Let $a^2$ and $b^2$ be the representation of the two vectors in a second frame $o_2x_2y_2z_2$ in $\R^3$, with the same origin as the first frame.

Let $R^1_2$ be the rotation matrix relating the two frames. Then we have 
\begin{align*} 
a^1 = R^1_2 a^2 \\ 
b^1 = R^1_2 b^2
\end{align*}

Writing the dot product equation we have, 
\begin{align*} 
a^1 \cdot b^1 &= (a^1)^T b^1 = (R^1_2 a^2)^T  R^1_2 b^2 = (a^2)^T (R^1_2)^T  R^1_2 b^2 \\
&= (a^2)^T b^2 = a^2 \cdot b^2 
\end{align*}
where we used the fact that $(R^1_2)^T  R^1_2 =I$

Hence, the dot product is independent from the choice of the reference frames.

\subsection*{2-2}

The length of a vector $v$ is given by the 2-norm, $\|v\|_2 = \sqrt{v^Tv}$.\\~

Using the above, we have $\|Rv\|^2_2=v^TR^TRv=v^Tv=\|v\|_2^2$, where it was used that $R^TR=I$ for an orthogonal / rotation matrix.

\subsection*{2-3}

By denoting $d=p_1-p_2$, the relation becomes $\|d\|_2=\|Rd\|_2$, which is the same as the one in the exercise \textbf{2-2}


